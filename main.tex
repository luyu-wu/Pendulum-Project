\documentclass[prl,twocolumn,amsmath,amssymb,superscriptaddress]{revtex4-2}

\usepackage{graphicx}
\usepackage{verbatim}
\usepackage{braket}
\usepackage{epsfig}
\usepackage{epstopdf}
\usepackage{amsfonts}
\usepackage{amsthm}
\usepackage{amsmath}
\usepackage{amssymb}
\usepackage{color}
\usepackage[dvipsnames,svgnames,table]{xcolor}
\usepackage{hyperref}
\hypersetup{colorlinks=true,linkcolor=NavyBlue,citecolor=BrickRed,urlcolor=NavyBlue}
\usepackage{dsfont}
\usepackage{color}
\usepackage{grffile}
\usepackage{bm}
\usepackage{lipsum}
%end of packages

\begin{document}
\title{PHY180: Pendulum Project}
\author{Luyu Wu Vankerkwijk}
\vspace{40pt}

\date{\today}


\maketitle

\section{Introduction}
This report investigates the behavior of a pendulum through experimental analysis, and correlates the findings to the given equation of the pendulum:
\vspace{-4pt}
\begin{equation}
    \theta(t) = A_0e^{-t/\tau}\cos\left(\frac{2\pi t}{T}+\phi_0\right)
    \label{eq:given}
\end{equation}

This equation (Wilson, 2025) solves the angle of the pendulum as a function of time, where $A_0$ and $\phi_0$ are initial conditions, and $\tau$ a constant damping factor. T (the period) is given as (Wilson, 2025):

\begin{equation}
    T=2\sqrt{L}
    \label{eq:Period}
\end{equation}

Experiments were performed using a custom-built, length-adjustable pendulum, with its motion tracked using a camera. Data was processed to determine experimental relationships between amplitude, period, damping, and length.

The period was found to be quadratically related to the angle of oscillation. For $L=0.75\,$m, the period was fit quadratically against amplitude ($T=aA^2+bA+c$), values $a=0.109\pm0.0007$, $b=0.000703\pm0.0005$, and $c=1.73\pm0.0005$ were found. This contradicts the given Eq. \ref{eq:Period}, which implies $T$ is not proportional to amplitude, whereas we find it to be quadratically proportional.

A small-angle regime where the governing equation is experimentally valid (amplitude dependent terms experimentally disappears beneath error) was found to be $-0.15\,$radians$<A<0.15\,$radians.

Investigating the effects of pendulum length, we measured the period of the pendulum against length within the small-angle regime. By power fitting length against period data ($T=aL^b$), fitting constants $a=2.00\pm0.02$ and $b=0.51\pm0.01$ were found. Thus, Eq. \ref{eq:Period} is verified to be valid for small amplitudes.

The damping of the system is also investigated as the dimensionless Q-factor ($Q=\pi\frac{\tau}{T}$). The exponential fitting of $\tau$ and Q-counting methods (Wilson, 2025) were used and gave results in error of each other ($202\pm0.9$ and $200\pm20$ respectively) for $L=0.75\,$m. This implies the damping shape is close to exponential at low angles.

From lengths varying between $0.15\,$m$\rightarrow 0.75\,$m, the Q-factor was quadratically fit against length ($Q = aL^2 + bL + c$). Values $a=-480\pm60$, $b=550\pm60$ $c=49\pm10$ were found for my setup. This physically implies a length where the pendulum may oscillate the most times before stopping.

\newpage
\color{gray}

\section{Experimental Methods}
The experimental setup consists of a 3m string, wrapped around a clothes hanger attached to an overhang. A cloth hanger is used such that the string can be spooled to adjust the length of the pendulum. An image of the setup can be seen in Fig. \ref{fig:experiment_setup}.

The top of the string is bound tightly. At the bottom of the pendulum, a metal mass with a ring soldered onto it is tied. \color{Black}The mass of the pendulum blob is $54\pm0.5\,$g while the full string's is $2.0\pm0.5\,$g.\color{Gray} A strand length of $74.5\pm0.05\,$cm was used in trials where length was kept constant. A calibration ruler is placed on the hanger to provide a reference for tracker.

The pendulum was carefully released by hand. Out-of-plane oscillations were analyzed to ensure minimal effect (see Appendix B).

\begin{figure}[htb]
    \includegraphics[width=0.6\linewidth]{setup.png}
    \caption{\color{Gray}Picture of experimental setup. Pendulum blob is visible in the lower left corner, while the DSLR (Fujifilm XT-30) can be seen in the right. OSP Tracker software used to obtain projected pendulum position as a function of time.}
    \label{fig:experiment_setup}
\end{figure}


The pendulum system was recorded from approximately 4 meters away using a DSLR camera (to minimize parallax). A shutter speed of $1/1000$ and motion FPS of $60$ were used to minimize motion blurring, and ensure a high time resolution (roughly 100 frames/oscillation).

All data was first tracked in \textit{OSP Tracker} before being processed in a custom Python script (see Appendix A). In processing the data, we assume that the motion of the pendulum is continuous (for interpolation between points) to achieve better time resolution.

SciPy peak detection was performed to find local maxima. Periods were determined from the time difference between adjacent peaks ($T_n=t_{n+1}-t_{n}$). 'Negative amplitudes' were determined from troughs.

\section{Analysis}


\subsection{Pendulum Period and Amplitude}

First, the period of the system is analyzed. The given equation (Eq. \ref{eq:Period}) assumes the period of the system to be a constant in relation to amplitude. By analyzing the relationship between amplitude and period experimentally, we can determine if this assumption is correct.

\color{black}

In Fig. \ref{fig:amplitude-period},  we plot the transient amplitude of the pendulum vs. the transient period. We see the data strongly fits a quadratic, where $[T=aA^2+bA+c]$; fitting constants $a=0.109\pm0.0007$, $b=0.000703\pm0.0005$, and $c=1.73\pm0.0005$ are found. Noticeably, the quadratic and constant terms are experimentally significant, while the linear term ($b$) is experimentally 0 (within 2 standard deviations of 0). This contradicts Eq. \ref{eq:Period}, which implies $T$ is not proportional to amplitude, whereas we observe it to be quadratically proportional.

\begin{figure}[htb]
    \hspace{-20pt}
    \includegraphics[width=0.85\linewidth]{amp_per_error_of_mean.png}
    \caption{\color{Gray}Plot of pendulum period against oscillation amplitude. Datapoint errorbars are calculated as the standard deviation of the mean. \color{Black}In the legend, $A$ is the transient amplitude, while $a,b,c$ are quadratic fitting constants.}
    \label{fig:amplitude-period}
\end{figure}
\color{gray}
Subsequent trials were done in amplitude regimes where period is experimentally constant. This correlates to $-0.15<A<0.15$, where the linear approximation is within error for all datapoints (see Fig. \ref{fig:small-angles}).

\begin{figure}[htb]
    \hspace{-20pt}
    \includegraphics[width=0.8\linewidth]{Q-factor-smallangleusage.png}
    \vspace{-8pt}
    \caption{\color{Gray}Plot of pendulum period against oscillation amplitude at small angles. Independent of data shown in Fig. \ref{fig:amplitude-period}.}
    \label{fig:small-angles}
\end{figure}
\newpage
\subsection{Damping Factor}

Investigating the damping is a critical part of understanding the pendulum's motion. We can use the $Q$-factor, which defines the ratio of energy loss in relation in a radian of a cycle.

$Q$-factor can be found through an exponential fit of $\theta$ to find $\tau$, where $Q=\pi\frac{\tau}{T}$. The real pendulum system experiences various damping forces which vary as amplitude decreases. This causes the Q-factor to be non-constant across amplitude in our system as visible in Figure \ref{fig:decay}. This disagrees with the given equation which gives $\tau$ as a constant.

\begin{figure}[htb]
    \hspace{-24pt}
    \includegraphics[width=0.8\linewidth]{decay.png}
    \caption{Plot of pendulum amplitude against time. Two $Q$-factors are found for different regions where $t_1\in[0s,25s], t_2\in[75s,125s]$. \color{Gray}This shows $Q$/$\tau$ to be amplitude-dependent.}
    \label{fig:decay}
\end{figure}

If $Q$ is instead found at low-amplitudes, where the small-angle approximation is more appropriate, we observe a much better fit (Fig. \ref{fig:decay_small_angle}).
\begin{figure}[htb]
    \includegraphics[width=0.8\linewidth]{low_angle_q.png}
    \caption{\color{Gray}Plot of pendulum amplitude against time. Since the range of the amplitudes is smaller in this case, an exponential fit is much better.}
    \label{fig:decay_small_angle}
\end{figure}

The Q-factor is found to be $202\pm0.9$ using the fit $\tau$ factor. Error is found through the fit error.

Alternatively, the $Q$-factor can be found by counting the oscillations it takes for the system to dampen to $A_0(e^{-\pi/C})$, then multiplying that number by $C$ (Wilson, 2025). A visualization of this method can be seen in Fig. \ref{fig:count_q}.

\begin{figure}[htb]
    \hspace{-20pt}
    \includegraphics[width=0.9\linewidth]{count_decay.png}
    \caption{\color{Gray}Amplitude of oscillation vs. oscillation number. Oscillation counting method is used to find number of oscillations before amplitude reaches $e^{-\pi/4}$ (the red line). The blue range represents amplitude peaks which intersect the red line. Here, the number of oscillations was found as $50\pm5$, thus Q-factor is $200\pm20$.}
    \label{fig:count_q}
\end{figure}

Using oscillation counting, the Q-factor is found to be $200\pm20$, where the larger error as compared to the exponential fit can be explained by the more discretized methodology.

\color{black}

Importantly, the Q-factors found by the two different methods are highly similar and within error of each other ($202\pm0.9$ and $200\pm20$ respectively). This implies the damping shape in the low-amplitude regime to be close to exponential, making Q-factor a valid characterization during low-amplitude oscillations. Moving forwards, we use $\tau$ (exponential) fitting, as the error is lower.
\newpage


\subsection{Length vs. Period}

Another area is investigating the effect of changing the string length on pendulum period.

To obtain insight, we can find the period of pendulums of different lengths by recording videos for each pendulum length and applying the same processing used before to find period.

During the experiment, each video was composed of at least 100 oscillations, and analyzed a range of oscillations $A < 0.15$ such that the period was experimentally constant in relation to amplitude changes.

\begin{figure}[htb]
    \hspace{-20pt}
    \includegraphics[width=0.9\linewidth]{length-period.png}
    \caption{Log-Log Oscillation Period vs. Length of Pendulum. Each datapoint here represents a distinct tracked movie, where the error is the standard error of the mean. A linear scale plot, as per instructions, can be found in Appendix E.}
    \label{fig:lengthperiod}
\end{figure}

In Fig. \ref{fig:lengthperiod}, we see that datapoints, when plotted on log-log axis, form a straight line. Applying a power fit where $[T=kL^n]$, we get $k=2.00\pm0.02$ and $n=0.51\pm0.01$, both of which are within error of the predicted period, $T=2\sqrt{L}$ (Eq. \ref{eq:Period}).

This experimentally shows the pendulum's period follows a root ($n=0.5$), or near root, proportionality with length, as suggested by the equation. Additionally, the experimental coefficient $k=2.00\pm0.02$ matches well with the given $k=2$ value. This implies the given equation accurately predicts period at low-amplitudes.

As additional insight, we notice the dimensional inconsistency in the given equation (stating $m^{\frac{1}{2}}=s$). If we instead solve the ODE using the small-angle approximation, we obtain $T=2\pi\sqrt{\frac{L}{g}}$. Taking $g$ out of the root, and multiplying it by $2\pi$ we find $T=[2.0061\frac{s}{m^{\frac{1}{2}}}]\sqrt{L}$. This explains why Eq. \ref{eq:Period} gives such a close approximation while being dimensionally inconsistent.

\newpage
\subsection{Length vs. Q-Factor}

Next, we investigate the Q-factor of the pendulum in relation to the length. The Q-factor equation is given as $Q=\pi\frac{\tau}{T}$ (Wilson, 2025), meaning it is inversely proportional to period. However, as we do not know how $\tau$ is dependent to length (it is unique to each system), a theoretical prediction of the trend would fall outside the scope of this lab.

Our experimental data is plotted in Fig. \ref{fig:qfactperiod} with a quadratic fit line. Datapoints are found with $\tau$ fitting, as it produces lower error. For $Q=aL^2+bL+c$, fitting constants $a=-480\pm60$, $b=550\pm60$, $c=49\pm10$ are found. Qualitatively, we see the Q-factor to increase with length to around $L=0.6\,$m before starting to decrease. Physically, this means that the pendulum stops after less oscillations at small lengths and large lengths, with some sweet spot in the middle.
%As we have confirmed T to be positively correlated with L (see fig. \ref{fig:lengthperiod}), this implies the Q-factor to be negatively correlated with length (given $\tau$ is constant).

\begin{figure}[htb]
    \hspace{-20pt}
    \includegraphics[width=0.9\linewidth]{length-Q.png}
    \caption{Q-Factor vs. Length of Pendulum. Each datapoint here represents a distinct tracked movie, where the error is the error of the mean. Results using Q-counting instead of $\tau$ fitting can be seen in Appendix F.}
    \label{fig:qfactperiod}
\end{figure}


It should be understood that this trend is not meant to be extrapolated, and instead only shows the non-linearity of this relationship. Extrapolating the quadratic trend suggests that Q decreases below 0 at a longer length, which is inappropriate. Additionally, the uncertainty of the quadratic fit is quite high, implying the actual relationship is unlikely to be quadratic.
%\vspace{120pt}
%
\color{gray}
\section{Determining Uncertainties}

Uncertainties were estimated by taking the larger of Type A or Type B evaluations. This section largely explains the derivation of Type B uncertainties, as they are not immediately clear in certain cases. Meanwhile, Type A uncertainties are simply estimated through standard deviation of the mean through multiple trials.

The amplitude uncertainty are estimated through Type B uncertainty. As OSP Tracker may not always select the centre of of the pendulum blob, this may cause measurement errors. It is estimated as $\epsilon = \frac{w}{2(l)}$ for a 68\% confidence. Measurement of the width was done in Tracker to calibrate error to the data (see Appendix C). This error may be minimized in the future by using an indicator (e.g. black dot) instead of tracking the whole blob, however, there may be issues ensuring the pendulum does not rotate around its radial axis. Error in the length of the pendulum is similarly obtained as it is also calculated through Tracker.

The Type B uncertainty in the time of measurement (which is the same as period) is negligible as the camera's shutter speed is $1/1000$ and frame timing concerns are negligible. Error in period-amplitude plots is thus calculated through Type A uncertainty. These are predominantly affected by Type-B amplitude uncertainty (see Appendix C), which through peak detection, translates into noise in period measurement.

Fit constant errors are found by diagonalizing the fit's covariance matrix and taking the square root. The error in the fit is propagated to values which depend on it (namely the Q-factor which depends on the fitted $\tau$).

\color{Black}
\section{Conclusion}
In this report, various aspects of the pendulum's motion were investigated in relation to the given equations.

An experimental setup consisting of a metal-blob pendulum and DSLR camera was constructed. Precautions were taken to ensure error due to string mass, parallax, and out-of-plane oscillations were minimized.

Using experimental data, we observe that $T$ varies as amplitude changes, unlike what the equation suggests, but instead quadratically proportional to the amplitude of oscillation (see Fig. \ref{fig:amplitude-period}, \ref{fig:decay}).

By analyzing oscillations at small angles (within error of a constant period), we are able to characterize the Q-factor of the system (see Fig. \ref{fig:decay_small_angle}).  This value was verified to be within error of the manual oscillation counting method as well.

The period-length equation (Eq. \ref{eq:Period}) was found to be valid in small angles and its dimensional inconsistency was explained.

Lastly, a roughly quadratic trend was found between the Q-Factor of the system at low amplitudes and the length of the pendulum within $0.2\,$m$<L<0.8\,$m.

In general, it was shown that the equations given are valid in the low amplitude regime of the system.

\color{black}

\onecolumngrid

\newpage
\section{Appendix}

\subsection{A. Processing Code}

All of the code (as well as video files and tracked CSV) used for this project is open-sourced on GitHub, and can be accessed here: \href{https://github.com/luyu-wu/Pendulum-Project}{https://github.com/luyu-wu/Pendulum-Project}.

All code is written by me, with credit to modules NumPy, MatPlotLib, and SciPy.

\subsection{B. Out-of-plane Oscillations}

In the case that this report investigates, the pendulum has 2 degrees of freedom ($\phi, \theta$). Since the plane that our camera projects is not sensitive to $\phi$, the resultant $\theta$ found is dependent on $\phi$ through $\cos(\phi)x_{pendulum} = x_{observed}$.

Thus, tracking the pendulum using a camera, care must be taken to avoid the pendulum going out of plane (in other words, keep $\phi$ at a desirable constant value).

A solution to this is using purely the $y$-tracked values, as these are not dependent on $\phi$ (ignoring parallax, which is not important in our setup).
However, at small angles, these are significantly harder to track due to vanishing derivative $\frac{dcos(\theta)}{d\theta}$ as $\theta \Rightarrow 0$.

In the future, to improve this, a pendulum using two strings could be made, such as to crudely restrict it to one DoF.

\begin{figure}[htb]
    \includegraphics[width=0.4\linewidth]{out-of-plane.png}
    \label{fig:out-of-plane}
    \caption{This graph shows the differences in results between tracking with $\phi$ dependence and without. Note that the reason no negative $\theta$ exists is because that would require us to know $\phi$, something not possible purely from $y$.}
\end{figure}

\subsection{C. Measurement Uncertainty of Amplitude}

As mentioned in \textbf{Determining Uncertainties}, the Type B uncertainty of the pendulum's position must be determined. To find this, the maximum width of the pendulum in the Tracker software is found. This is at the bottom of its motion, where motion blur is greatest. The uncertainty is then found through this via $\epsilon = \frac{w}{2(l)}$ so as to represent a 68\% confidence interval. Note that this assumes Tracker has a uniform selection density across the pendulum body, which implies this error is overestimated. This process can be seen in Fig. \ref{fig:body}.
\begin{figure}[htb]
    \includegraphics[width=0.2\linewidth]{pendulum-body.png}
    \caption{The measurement of the width of the pendulum is shown here. The length is propagated from the calibration stick.}
    \label{fig:body}
\end{figure}

\subsection{D: Q-Factor of Different Lengths}
\begin{figure}[htb]
    \vspace{-16pt}
    \includegraphics[width=0.6\linewidth]{fitting.png}
    \caption{Here, data from different length trials are plotted together. Note the Y-axis is in log-scale such that the data is linearized.}
    \label{fig:fits}
    \vspace{-20pt}
\end{figure}


%\subsection{E: Preliminary Analysis for $\tau$ Dependence}

%Starting with the equation for viscous damping:
%\begin{equation}
%    F_d = -\frac{dr}{dt}c_d
%\end{equation}
%Where r is an arbitrary displacement vector.

%We can write equation of motion with viscous drag in polar coordinates (by solving torque and moment of inertia):
%\begin{equation}
%    \ddot{\theta} = -sin(\theta)\frac{g}{l}+\dot{\theta}l^2\frac{c_d}{ml^2}
%\end{equation}
%We assume $c_d$ to be constant (as it should be a laminar drag dominated effect).

%First, taking the small angle approximation, we obtain:
%\begin{equation}
%    \ddot{\theta} = -\theta\frac{g}{l}+\dot{\theta}lc_d
%\end{equation}

\subsection{E: Non-Log Plot for Length vs. Period}

\begin{figure}[htb]
    \vspace{-16pt}
    \includegraphics[width=0.3\linewidth]{non-log-period.png}
    \caption{Non-log plot for Length vs. Period as required by instructions.}
    \label{fig:qfact_counting}
\end{figure}

\subsection{F: Oscillation Counting for Length vs. Q-Factor}
\begin{figure}[htb]
    \vspace{-16pt}
    \includegraphics[width=0.5\linewidth]{length-q-counting.png}
    \caption{Q-Factor vs. Length of Pendulum. Q-Counting error bars are not shown as these values are unused, and only serve to verify the rough accuracy of the $\tau$ fitting method.}
    \label{fig:qfact_counting}
\end{figure}

\newpage

\section{References}

Wilson, Brian, “PHY180 Pendulum Project”, from
\href{https://q.utoronto.ca/courses/411727/files/39071655?module_item_id=7122439}{q.utoronto.ca}, 2025.


\subsection{AI Statement}

No form of AI was used in planning the lab, writing the lab, or writing the processing code.

AI was used to spell-check the report after finishing with the prompt "Note any typing errors". 6 minor suggestions regarding spelling mistakes were accepted. I decided to use it because there is no spell checker available in my IDE for LaTeX.

\end{document}
